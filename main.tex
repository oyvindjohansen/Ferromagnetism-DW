\documentclass[1p]{elsarticle}		% 5p gir 2 kolonner pr side. 1p gir 1 kolonne pr side.
\journal{students}
\usepackage[T1]{fontenc} 						% Vise norske tegn.
% \usepackage[latin1]{inputenc}		% Velger tengsettet i dette dokumentet			
\usepackage[english]{babel}		
\usepackage[utf8]{inputenc}
\usepackage{graphicx}
\usepackage{hyperref}
\usepackage{amsmath,amssymb}
\usepackage{esint}
\usepackage[output-decimal-marker = {,}]{siunitx}
\usepackage[font=footnotesize,labelsep=period,labelfont=bf,margin=1cm]{caption}
\usepackage{import}
\usepackage{caption}
\usepackage{subcaption}


\setcounter{totalnumber}{5}
\renewcommand{\textfraction}{0.05}
\renewcommand{\topfraction}{0.95}
\renewcommand{\bottomfraction}{0.95}
\renewcommand{\floatpagefraction}{0.35}
\renewcommand{\d}[1]{\ensuremath{\operatorname{d}\!{#1}}}
\setlength\parindent{0pt}



\setcounter{totalnumber}{5}
\renewcommand{\textfraction}{0.05}
\renewcommand{\topfraction}{0.95}
\renewcommand{\bottomfraction}{0.95}
\renewcommand{\floatpagefraction}{0.35}

\makeatletter
\def\ps@pprintTitle{%
  \let\@oddhead\@empty
  \let\@evenhead\@empty
  \let\@oddfoot\@empty
  \let\@evenfoot\@oddfoot
}
\makeatother

\begin{document}

\title{Magnetization dynamics of domain walls and skyrmions in micromagnetics}
\author{\O yvind Johansen}
\date{\today}

\maketitle
\tableofcontents

\section{Introduction}

\section{Energy terms in micromagnetics}
\subsection{Anisotropic energy}
Magnetic materials may be anisotropic, meaning that certain directions of the magnetization in the material are more energetically favorable than others. This stands in contrast to isotropic materials where the energy is independent of the direction of the magnetization, assuming uniform magnetization. There are several possible causes for anisotropy in a material, the most common one being the crystalline structure in the material. One of the most important causes of magnetocrystalline anisotropy is the spin--orbit interaction. This interaction arises from the fact that in the restframe of the electron, the proton is orbiting the electron, thereby creating a varying electric field. Since the electric field is varying, Maxwell's equations say that there will also be a corresponding magnetic field. This magnetic field is proportional to the angular momentum $\vec{L}$ of the electron. The electron also has a magnetic moment $\vec{\mu}$, proportional to its spin $\vec{S}$, that will interact with the magnetic field. The energy is lowest when the magnetic moment and field are aligned, which we will see when discussing the Zeeman energy later in the text. Combining the spin--orbit interaction with the orientation of the atoms in the lattice it is possible that some directions in the material are magnetized easier than others. This is magnetocrystalline anisotropy. The shape of the magnetic particles may also cause anistropy in the material, which will be briefly discussed in the section on the demagnetization energy. 

In anisotropic magnetic materials one defines different axes; the easy, intermediate and hard axes. If the magnetization is oriented along the easy axis, the anisotropic energy is at its minimum value. Similarly, if the magnetization is oriented along the hard axis the anisotropic energy is at its maximum value. The intermediate axis is the direction along which there is a saddlepoint in the anisotropic energy. In most materials the anisotropic energy is the same for both possible orientations of the magnetization along the axes. When this is the case, the terms in the anisotropic energy can only contain even powers of the components in the magnetization vector.

\subsubsection{Uniaxial anisotropy}
The simplest form of anisotropy is the case where there is only one axis that the anisotropic energy varies with. This is known as uniaxial anisotropy. This can often be found in crystal structures with a single principal axis, such as the hexagonal and simple tetragonal lattices. If we consider a uniform magnetization $\vec{M} = \left[M_x, M_y, M_z\right]$ and let one of the directions along the axis in the material that the anisotropic energy varies be the unit vector $\hat{n}$, the anisotropic energy must be on the form

\begin{align}
\label{eq:uniaxialanisotropy}
E_A = V \sum_{i=0}^{\infty} K_i (\frac{\vec{M}\cdot\hat{n}}{M_s})^{2i}
\end{align}

if there is to be no preferential direction with respect to energy along the axis. Here $V$ is the volume, $M_s = |\vec{M}|$ the saturation magnetization and $K_i$ are the anisotropy constants. By only considering the lowest order in the anisotropic energy it becomes

\begin{align}
\label{eq:uniaxialanisotropy}
E_A = K V (\frac{\vec{M}\cdot\hat{n}}{M_s})^2
\end{align}

by ignoring the constant energy term as that only leads to a constant shift in the energy. If $K < 0$ the axis that $\hat{n}$ points along is then the easy axis, and if $K > 0$ the axis is the hard axis. Note that if we have a non-uniform magnetization we let

\begin{align*}
\vec{M} &\rightarrow \vec{M}(\vec{r}), \\
V &\rightarrow \int \d V.
\end{align*}

\subsubsection{Cubic anisotropy}
Many magnetic materials, such as iron, have cubic crystal structures. This can give rise to cubic anisotropy in the magnetic material. To describe cubic anisotropy it can be useful to introduce directional cosines that give the component of a unitary vector along the $x$-, $y$- and $z$-axes. Using spherical coordinates, these directional cosines are defined to be $\alpha = \sin\theta\cos\phi$, $\beta = \sin\theta\sin\phi$, $\gamma = \cos\theta$. Note that the directional cosines can also be written in terms of the magnetization, so that $\alpha = M_x/M_s$, $\beta = M_y/M_s$ and $\gamma = M_z/M_s$. If there is to be no preferential direction with respect to energy along the crystal axes, the anisotropic energy can only contain even powers of $\alpha$, $\beta$ and $\gamma$. In addition, to have cubic symmetry in the anisotropic energy, the energy must be invariant under any interchanges among the directional cosines \cite{Kittel:ISSP}. The first term in the anisotropic energy that depends on the directional cosines will therefore be proportional to $\alpha^2+\beta^2+\gamma^2$, but using the definitions of the directional cosines we find that

\begin{align}
\label{eq:cosinesconstraint}
\alpha^2+\beta^2+\gamma^2 = \frac{M_x^2+M_y^2+M_z^2}{M_s^2} = \frac{|\vec{M}|^2}{|\vec{M}|^2} = 1.
\end{align}

This term is therefore just a constant, so to find a term that varies with the directional cosines we must go to fourth order:

\begin{align}
\label{eq:cubicanisotropy}
E_A = E_A^{(0)} + \int (K_1 (\alpha^2\beta^2+\alpha^2\gamma^2+\beta^2\gamma^2) + \ldots ) \d V.
\end{align}
The sixth order term would be on the form $K_2 V \alpha^2\beta^2\gamma^2$ and so on. If one looks at the form of the energy in \eqref{eq:cubicanisotropy} one can determine the easy and hard axes in the material. The integrand has to be minimized along the easy axes, and maximized along the hard axes. If one only considers the fourth order terms in the energy, we have to minimize and maximize the function $K_1 (\alpha^2\beta^2+\alpha^2\gamma^2+\beta^2\gamma^2)$ with the constraint given by \eqref{eq:cosinesconstraint}. Using Lagrange multipliers, this breaks down to solving 

\begin{align}
\nabla f(\alpha, \beta, \gamma) &= \lambda \nabla g(\alpha, \beta, \gamma), \\
g(\alpha, \beta, \gamma) &= 0,
\end{align}
with
\begin{align}
f(\alpha, \beta, \gamma) &= K_1 (\alpha^2\beta^2+\alpha^2\gamma^2+\beta^2\gamma^2), \\
g(\alpha, \beta, \gamma) &= \alpha^2+\beta^2+\gamma^2 - 1,\\
\nabla &= \left[\frac{\partial}{\partial\alpha}, \frac{\partial}{\partial\beta}, \frac{\partial}{\partial\gamma}\right].
\end{align}
This gives us the following set of equations to solve:

\begin{align*}
2K_1\alpha (\beta^2+\gamma^2) &= 2\alpha\lambda, \\ 
2K_1\beta (\alpha^2+\gamma^2) &= 2\beta\lambda, \\ 
2K_1\gamma (\alpha^2+\beta^2) &= 2\gamma\lambda, \\ 
\alpha^2+\beta^2+\gamma^2 - 1 &= 0.
\end{align*}

One can easily see that one solution of the equations is given by $\lambda = 0$, one of the directional cosines being $\pm 1$ and the other two being 0. Inserting this solution into $f(\alpha, \beta, \gamma)$ we get $f = 0$. \\

Assuming only one of the directional cosines is 0, for example $\gamma$, one gets

\begin{align*}
K_1 \beta^2 &= \lambda, \\ 
K_1 \alpha^2 &= \lambda.
\end{align*}

Using the constraint in \eqref{eq:cosinesconstraint} one finds $\lambda = K_1/4$, and both non-zero directional cosines being $\pm 1/2$. Inserting this solution into $f(\alpha, \beta, \gamma)$ we get $f = K_1/4$. \\

Assuming none of the directional cosines are 0, one gets

\begin{align*}
K_1 (\beta^2+\gamma^2) &= \lambda, \\ 
K_1 (\alpha^2+\gamma^2) &= \lambda, \\ 
K_1 (\alpha^2+\beta^2) &= \lambda.
\end{align*}

Using the constraint in \eqref{eq:cosinesconstraint} one finds $\lambda = 2K_1/3$, and each of the directional cosines being $\pm 1/\sqrt{3}$. Inserting this solution into $f(\alpha, \beta, \gamma)$ we get $f = K_1/3$. This means that if $K_1 > 0$ the energy is minimized by the first solution, making the $x$-, $y$- and $z$-axes the easy axes. The hard axes are then given by the third solution which corresponds to the four axes $x = \pm y = \pm z$, $x = \pm y = \mp z$. If $K_1 < 0$ the energy is maximized by the first solution and minimized by the third, so that the easy and hard axes switch from the case when $K_1 > 0$. The second solution denotes a local minimum if $K_1<0$ and a local maximum if $K_1>0$. The second solution does not give any intermediate axes, as there is no saddle point in the energy.


\subsection{Demagnetization energy}
The magnetic dipoles inside a ferromagnet will generate a magnetic field, known as the demagnetizing field. This name stems from the behavior of the field, as it will try to reduce the total magnetic moment in the magnet. This is because a high magnetic moment in the magnet will generate a strong external magnetic field, which costs energy. Magnetic materials consist of magnetic dipoles. If the magnetization is uniform in the magnet there will be no poles inside the magnet, as the positive and negative poles cancel each other out. There will, however, be a positive pole at one surface of the magnet and a negative pole at the opposing surface that do not cancel out. This causes a large total magnetic moment in the magnet, as it is proportional to the length between the poles. To minimize the total magnetic moment and thereby the energy, the demagnetizing field will therefore create magnetic domains, where the different domains have uniform magnetization in different directions. There is a limit to how small the domains can be, as the exchange interaction will begin to dominate on short ranges, which we will discuss later. 

The energy in a ferromagnet will depend on the orientation of the magnetic dipoles with respect to the demagnetizing field created by the other magnetic dipoles. This demagnetization energy is given by

\begin{align}
\label{eq:demagenergy}
E_D = -\frac{\mu_0}{2}\int \d {^3}r \vec{M}(\vec{r})\cdot\vec{H}_D(\vec{r}),
\end{align}

with $\vec{H}_D$ being the demagnetization field. From this equation it looks like the demagnetizing field will attempt to align the magnetization with itself, which is not the behavior that was discussed earlier. One can not make this conclusion yet, however, as the demagnetizing field is at this point an unknown function of the magnetization $\vec{M}$. To determine the demagnetization energy one needs to find the magnetic field generated by a magnetic dipole $\vec{M}$. The vector potential of a magnetic dipole is known to be

\begin{align}
\vec{A}(\vec{r}) = \frac{\mu_0}{4\pi}\frac{\vec{M}\times\vec{r}}{r^3}.
\end{align} 

The magnetic induction $\vec{B}(\vec{r})$ is defined as

\begin{align}
\vec{B}(\vec{r}) = \nabla \times \vec{A}(\vec{r}).
\end{align}

Using the Einstein summation convention and the Levi--Civita tensor $\varepsilon_{ijk}$ one can then find the magnetic induction generated by a single magnetic dipole:

\begin{align*}
B_i &= \varepsilon_{ijk} \partial_j A_k \\
&= \varepsilon_{kij}\varepsilon_{klm} \frac{\mu_0}{4\pi}\partial_j\frac{M_l r_m}{r^3} \\
&= (\delta_{il}\delta_{jm}-\delta_{im}\delta_{jl})\frac{\mu_0}{4\pi}(\frac{M_l\delta_{jm}}{r^3}-\frac{3M_lr_mr_j}{r^5}) \\
&= \frac{\mu_0}{4\pi}(\frac{3(\vec{M}\cdot\vec{r})r_i}{r^5}-\frac{M_i}{r^3}).
\end{align*}

This means that the magnetic field strength of the demagnetization field generated by the entire magnet at a position $\vec{r}$ is

\begin{align}
\label{eq:demagfield}
\vec{H}_D = \frac{1}{4\pi} \int \d {^3}r' (\frac{3(\vec{M}(\vec{r'}) \cdot (\vec{r}-\vec{r'})) (\vec{r}-\vec{r'})}{|\vec{r}-\vec{r'}|^5}-\frac{\vec{M} (\vec{r'})}{|\vec{r}-\vec{r'}|^3}).
\end{align}

\begin{figure}[h!]
\begin{center}
\includegraphics[width=0.8\textwidth]{Figures/dipole_field.pdf} 
\caption{The magnetic field generated by a single magnetic dipole located at the origin pointing in the $y$-direction. Broad field lines indicate a strong magnetic field.}
\label{fig:dipole_field} 
\end{center}
\end{figure}

In Figure \ref{fig:dipole_field} one can see that the demagnetizing field has a tendency to point in a different direction than the magnetization. At a position that is perpendicular to the magnetization the field is completely anti-aligned with the magnetization. Inserting \eqref{eq:demagfield} into \eqref{eq:demagenergy} one finds that the energy is

\begin{align}
\label{eq:demagenergy_mag}
E_D = \frac{\mu_0}{8\pi} \int \d {^3}r \int \d {^3}r' (\frac{\vec{M} (\vec{r}) \cdot \vec{M} (\vec{r'})}{|\vec{r}-\vec{r'}|^3} - \frac{3(\vec{M}(\vec{r}) \cdot (\vec{r}-\vec{r'})) (\vec{M}(\vec{r'}) \cdot(\vec{r}-\vec{r'}))}{|\vec{r}-\vec{r'}|^5}).
\end{align}

The energy is small when the total magnetization in the magnet is small, the demagnetizing field will therefore try to reduce the total magnetic moment in the magnet as discussed earlier. 

This integrand in \eqref{eq:demagenergy_mag} can be rewritten in a tensor form \cite{kruger2006current}. Using the relation

\begin{align}
\partial_i \frac{1}{|\vec{r}-\vec{r'}|} = -\frac{r_i-r_i'}{|\vec{r}-\vec{r'}|^3}
\end{align}

it is easy to see that

\begin{align*}
\partial_i\partial_j \frac{1}{|\vec{r}-\vec{r'}|} &= \partial_i(-\frac{r_j-r_j'}{|\vec{r}-\vec{r'}|^3}) \\
&= -\frac{\delta_{ij}}{|\vec{r}-\vec{r'}|^3}+3\frac{(r_i-r_i')(r_j-r_j')}{|\vec{r}-\vec{r'}|^5}.
\end{align*}

Defining the continuous demagnetization tensor to be

\begin{align}
N_{ij}(\vec{r}-\vec{r'}) = -\frac{1}{4\pi}\partial_i\partial_j \frac{1}{|\vec{r}-\vec{r'}|},
\end{align}

the demagnetization energy and field can then be written as

\begin{align}
E_D &= \frac{\mu_0}{2} \int \d {^3}r \int \d {^3}r' \vec{M}(\vec{r}) N(\vec{r}-\vec{r'})\vec{M}(\vec{r'}), \\
\vec{H}_D(\vec{r}) &= - \int \d {^3}r' N(\vec{r}-\vec{r'})\vec{M}(\vec{r'}).
\end{align}

Something worth noting is that if the demagnetization tensor $\bar{N}(\vec{r}-\vec{r'})$ is strictly diagonal, the energy can be written as

\begin{align}
E_D = \mu_0(N_{xx}M_x^2+N_{yy}M_y^2+N_{zz}M_z^2).
\end{align}

This is the case for ellipsoidal bodies \cite{kruger2006current}. This form of the demagnetization energy is similar to that of the uniaxial anisotropic energy in \eqref{eq:uniaxialanisotropy}. This means that ellipsoidal magnetic particles cause shape anisotropy, assuming the magnetic particle is not spherical ($N_{xx} = N_{yy} = N_{zz}$).


\subsection{Zeeman energy}
The Zeeman energy describes the energy from the interaction between the magnetization of the magnet and an external field $\vec{H}_{Z}$. The Zeeman energy is on the same form as the demagnetization energy in \eqref{eq:demagenergy}, with the internally generated demagnetization field replaced by the external magnetic field:

\begin{align}
\label{eq:zeemanenergy}
E_Z = -\mu_0\int \d {^3}r \vec{M}(\vec{r})\cdot\vec{H}_{Z}(\vec{r}).
\end{align}

The factor 1/2 in the demagnetization energy is to account for double counting of the dipole interactions, and is therefore not included in the Zeeman energy as the magnetic moments only interact with an external field. The Zeeman energy is at its minimum when the magnetization is aligned with the external field.

\subsection{Exchange energy}
Spins on a lattice will have an exchange energy with their nearest neighbors, as given by the Heisenberg Hamiltonian

\begin{align}
\label{eq:heisenberg}
H = -J\sum_{<i, j>} \vec{S}_i\cdot\vec{S}_j = -\frac{J S^2}{M_s^2}\sum_{<i, j>} \vec{M}_i\cdot\vec{M}_j,
\end{align}

with $J$ being the exchange integral and $\vec{S}_{i,j}$ the dimensionless spins. For a ferromagnet the lowest exchange energy is when the spins are aligned, meaning $J>0$. In the Hamiltonian we sum over each nearest neighbor pair $<i, j>$. To perform the sum we look at the magnetization in the continuum limit, with a slowly varying magnetization. It is then possible to do a Taylor expansion around the magnetization $\vec{M}_i$. If one only looks at a Taylor expansion in the $x$-direction, it can be written as

\begin{align}
\label{eq:taylormag}
\vec{M}_{i+1} \approx \vec{M}_i + a\frac{\partial \vec{M}_i}{\partial x} + \frac{a^2}{2}\frac{\partial^2 \vec{M}_i}{\partial^2 x},
\end{align}

with $a$ being the lattice constant and $\vec{M}_{i+1}$ the magnetization at the neighboring site to $\vec{M}_i$ along the $x$-axis. Similar expansions can be done in the $y$- and $z$-directions. Using this we can take a look at the contribution to the energy from $\vec{M}_i$ and its nearest neighbors in \eqref{eq:heisenberg}. For simplicity we consider a cubic lattice with a lattice constant $a$. Each lattice site has six nearest neighbors in three dimensions. Applying \eqref{eq:taylormag} to find the contributions to the energy from the nearest neighbors in the $x$-direction located at $\pm a \hat{x}$ relative to $\vec{M}_i$, we find

\begin{align*}
\vec{M}_i\cdot\vec{M}_{i-1}+\vec{M}_i\cdot\vec{M}_{i+1} &= 2\vec{M}_i\cdot\vec{M}_i + (a - a) \vec{M}_i\cdot\frac{\partial \vec{M_i}}{\partial x} + a^2\vec{M}_i\cdot\frac{\partial^2 \vec{M}_i}{\partial^2 x} \\
&= 2M_s^2 + a^2\vec{M}_i\cdot\frac{\partial^2 \vec{M}_i}{\partial^2 x}.
\end{align*}
Doing the same calculation for the nearest neighbors in the $y$- and $z$-directions we get that the exchange energy density from the magnetization at one site and its nearest neighbors is

\begin{align}
\epsilon_i = -\frac{JS^2}{a^3}(6+\frac{a^2}{M_s^2}\vec{M}_i\nabla^2\vec{M}_i).
\end{align}

To find the total exchange energy in the magnet we integrate the energy density over the volume of the magnet, so that

\begin{align}
\label{eq:exchangeenergylaplace}
E_E = -\int \d {^3}r \frac{JS^2}{2aM_s^2} \vec{M}(\vec{r})\nabla^2\vec{M}(\vec{r}).
\end{align}

The constant term in the energy has been ignored, and the factor $1/2$ has been introduced to account for the double counting of the interaction pairs. The energy in \eqref{eq:exchangeenergylaplace} can be rewritten to a more common form. Introducing the index notation and using the Einstein summation convention, we integrate \eqref{eq:exchangeenergylaplace} by parts: 

\begin{align}
\int \d {^3}r \vec{M}(\vec{r})\cdot\partial_i \partial_i \vec{M}(\vec{r}) = \int \d S \vec{M}(\vec{r})\cdot \frac{\partial}{\partial \hat{n}}\vec{M}(\vec{r}) - \int \d {^3}r \partial_i  \vec{M}(\vec{r})\cdot\partial_i \vec{M}(\vec{r}).
\end{align}

As $\vec{M}(\vec{r})$ is a vector of constant length, any change $\delta \vec{M}(\vec{r})$ must be perpendicular to $\vec{M}(\vec{r})$. This means that $\vec{M}(\vec{r})\cdot \frac{\partial}{\partial \hat{n}}\vec{M}(\vec{r}) = 0$, and the first term on the right hand side vanishes. The energy in \eqref{eq:exchangeenergylaplace} then becomes

\begin{align}
\label{eq:exchangeenergy}
E_E = \int \d {^3}r \frac{A}{M_s^2} ((\frac{\partial}{\partial x}\vec{M}(\vec{r}))^2+(\frac{\partial}{\partial y}\vec{M}(\vec{r}))^2+(\frac{\partial}{\partial z}\vec{M}(\vec{r}))^2),
\end{align}

where we have introduced the exchange constant $A$. As one can see the energy depends on the variation in the magnetization in all directions, and is independent of which direction the magnetization changes, it is only dependent on the magnitude by which it changes. Because the exchange constant $A$ is positive in a ferromagnet, the exchange energy is higher the more the magnetization varies. This can also easily be seen from \eqref{eq:heisenberg}, it costs energy when the spins are not aligned when $J>0$, which is the case in a ferromagnet. To minimize the exchange energy the magnetization therefore has to vary slowly. On short ranges this effect is much stronger than the demagnetization and magnetocrystalline anisotropy. This is why neighbouring spins have a tendency to align in a ferromagnet, while on longer ranges when the demagnetization effect becomes stronger than the exchange interaction the spins tend to be anti-aligned.

\section{Landau--Lifshitz--Gilbert equation}
The time evolution of the magnetization in a magnet is described by the Landau--Lifshitz--Gilbert (LLG) equation. To get an understanding of the origin of the equation, it is useful to look at the time evolution of spin, as the magnetic moment and therefore the magnetization is proportional to it. 

\subsection{Larmor precession}
The time evolution of an angular momentum $\vec{L}$ can be described by a variation on the classical Newton's second law for rotation:

\begin{align}
\label{eq:newton2rotation}
\frac{\textrm{d} \vec{L}}{\textrm{d} t} = \vec{T},
\end{align}

where $\vec{T}$ is the torque acting on the angular momentum. Spin is often compared to an intrinsic angular momentum of the electron, and will satisfy the same equation, with the exception that the vectors become operators as spin is a quantum mechanical effect. We are interested in looking at the magnetization on a microscopic scale, which would correspond to a semi-classical limit, so the quantum mechanical operators are replaced by their expectation values. 

A magnetic moment $\vec{\mu}$ will precess around an external magnetic field $\vec{H}$. This is known as Larmor precession, and is given by the torque

\begin{align}
\label{eq:larmortorque}
\vec{T} = \vec{\mu} \times \vec{H} = -\gamma \vec{S} \times \vec{H},
\end{align}

where we have introduced the gyromagnetic ratio $\gamma$. The gyromagnetic ratio gives the ratio of the magnetic moment relative to the spin of a particle. The minus sign is introduced because the magnetic moment and the spin of an electron point in opposite directions. For the electron the gyromagnetic ratio is given by

\begin{align}
\gamma = \frac{g_e\mu_B}{\hbar},
\end{align}

with the $g$-factor $g_e \approx 2$ and $\mu_B$ being the Bohr magneton

\begin{align}
\mu_B = \frac{e\hbar}{2m_e}.
\end{align}

By inserting \eqref{eq:larmortorque} into \eqref{eq:newton2rotation} and replacing the angular momentum with spin, we get

\begin{align}
\frac{\textrm{d} \vec{S}}{\textrm{d} t} =- \gamma \vec{S} \times \vec{H},
\end{align}

or by replacing the spin with the magnetization using that $\vec{S}$ is proportional to $\vec{M}$:

\begin{align}
\label{eq:mag_undamped}
\frac{\textrm{d} \vec{M}}{\textrm{d} t} = -\gamma \vec{M} \times \vec{H}.
\end{align}

\subsection{Effective field}

If the magnetic moments were free, meaning they did not interact with each other, the field $\vec{H}$ in \eqref{eq:mag_undamped} would just be the magnetic field in the material. The magnetic field is the sum of any external field and the internal demagnetizing field discussed earlier. The magnetic moments are not free, however, as we have effects such as exchange interaction between the magnetic moments and magnetic anisotropy. Landau and Lifshitz introduced an effective field $\vec{H}_{eff}$ that replaced the magnetic field $\vec{H}$ in \eqref{eq:mag_undamped} that would minimize the energy in the system at equilibrium \cite{LandauLifshitz1935}. The terms in the micromagnetic energy discussed earlier were the anisotropic energy, demagnetization energy, the Zeeman energy and the exchange energy. Using the results derived earlier, we can write the total energy in the system as

\begin{align}
\label{eq:micromagneticenergy}
E(\vec{M}) = \int \d {^3} r (\epsilon_A(\vec{M}) + \epsilon_D(\vec{M}) + \epsilon_Z(\vec{M}) + \epsilon_E(\vec{M})),
\end{align}

where the $\epsilon$'s are the energy densities of the different terms. In equilibrium we must require that $\delta E=0$, with $\delta E$ being defined by

\begin{align}
\label{eq:deltaE}
\delta E = \int_a^b \frac{\delta E[\vec{M}]}{\delta \vec{M}(\vec{r})}\cdot \delta \vec{M}(\vec{r}) \d {\vec{r}}.
\end{align}

The functional derivative $\frac{\delta E[\vec{M}]}{\delta \vec{M}(\vec{r})}$ is just the left hand side of the Euler-Lagrange equation, so that

\begin{align}
\label{eq:functionaldiff}
\frac{\delta E[\vec{M}]}{\delta \vec{M}(\vec{r})} = \frac{\partial E[\vec{M}]}{\partial \vec{M}(\vec{r})} - \frac{\partial}{\partial x} \frac{\partial E[\vec{M}]}{\partial (\frac{\partial \vec{M}}{\partial x})} - \frac{\partial}{\partial y} \frac{\partial E[\vec{M}]}{\partial (\frac{\partial \vec{M}}{\partial y})} - \frac{\partial}{\partial z} \frac{\partial E[\vec{M}]}{\partial (\frac{\partial \vec{M}}{\partial z})}.
\end{align}

As $|\vec{M}| = M_s$ is a constant, $\delta \vec{M}$ must be perpendicular to $\vec{M}$. Because of this, $\frac{\delta E[\vec{M}]}{\delta \vec{M}(\vec{r})}$ must be parallel to $\vec{M}$ for $\delta E$ to be 0 for arbitrary choices of $a$ and $b$. In \eqref{eq:mag_undamped} one can see that the system is in equilibrium if the effective field $\vec{H}_{eff}$ is parallel to $\vec{M}$, therefore the effective field must also be parallel to $\frac{\delta E[\vec{M}]}{\delta \vec{M}(\vec{r})}$. It is reasonable to assume that one of the terms in the effective field will just be the external magnetic field $\vec{H}_Z$. We can therefore look at $\epsilon_Z$, the energy density in \eqref{eq:zeemanenergy}, to find the proportionality constant. It is easy to see that by letting

\begin{align}
\vec{H}_{eff} = -\frac{1}{\mu_0}\frac{\delta \epsilon[\vec{M}]}{\delta \vec{M}(\vec{r})},
\end{align}

one of the terms in the effective field will be the external field $\vec{H}_Z$ (with $\epsilon = \frac{\partial E}{\partial V}$). By using this definition and the definition of the functional derivative in \eqref{eq:functionaldiff}, one then finds that the effective field is

\begin{align}
\label{eq:effectivefield}
\vec{H}_{eff}(\vec{r}) = \vec{H}_Z(\vec{r}) + \vec{H}_D(\vec{r}) + \frac{2A}{\mu_0M_s^2}\nabla^2\vec{M}(\vec{r}) -\frac{1}{\mu_0}\frac{\delta \epsilon_A[\vec{M}]}{\delta \vec{M}(\vec{r})}.
\end{align}

The last term depends on what kind of anisotropy there is in the material. For a uniaxial anisotropy as in \eqref{eq:uniaxialanisotropy}, the effective field term becomes 

\begin{align}
\label{eq:effielduniaxialani}
-\frac{1}{\mu_0}\frac{\delta \epsilon_A[\vec{M}]}{\delta \vec{M}(\vec{r})} = -\frac{2K_1}{M_s^2}(\vec{M}(\vec{r})\cdot\hat{n})\hat{n} - \frac{4K_2}{M_s^4}(\vec{M}(\vec{r})\cdot\hat{n})^3\hat{n}
\end{align}

up to fourth order in the directional cosine. For a cubic anisotropy the energy density $\epsilon_A$ becomes the integrand in \eqref{eq:cubicanisotropy} instead.

Looking at the effective field in \eqref{eq:effectivefield} we see that it consists of the external magnetic field and the demagnetizing field, as one would expect, and two effective field terms that originate from quantum mechanical effects. The exchange interaction between the magnetic moments cause an effective field that points in the direction that is the average rate of change in the magnetization $\vec{M}(\vec{r})$ at a point $\vec{r}$. If the magnetization is uniform or varies very slowly in space, the effective field from the exchange interaction is very small or non-existent. If the magnetization varies at a significant rate, the effective field from the exchange interaction will point in a direction that will try to decrease this rate of change, thereby minimizing the exchange energy. If one studies the effective field from the anisotropic energy, one finds that the field will try to align the magnetization with the easy axis, or move the magnetization away from the hard axis, depending on the axes in the material. For a uniaxial anisotropy, the effective field is described by \eqref{eq:effielduniaxialani}. We discussed earlier that the direction $\hat{n}$ was parallel to the easy axis if the anisotropy constants $K_i$ were $<0$, and it was parallel to the hard axis if the anisotropy constants were $>0$. Considering the case of $\hat{n}$ being parallel to te easy axis first, we see that the effective field will point in the direction that the component of $\vec{M}(\vec{r})$ has along the easy axis. This is regardless of which of the two directions the component points in, as we only have odd powers of $(\vec{M}(\vec{r})\cdot\hat{n})$ in the field. If $\hat{n}$ is parallel to the hard axis, the constants $K_i$ are positive, and the field points in the opposite direction that the component of $\vec{M}(\vec{r})$ has along the hard axis. The effective field then tries to reduce the component along the hard axis, to minimize the anisotropic energy.

\subsection{Damping and the Landau--Lifshitz equation}
The time evolution in \eqref{eq:mag_undamped} only describes the precession of the magnetization around a field $\vec{H}$. This system is not in equilibrium unless the magnetization is parallel to the field. In a physical system the magnetization will eventually relax to equilibrium. To model this, Landau and Lifshitz introduced a damping term that is perpendicular to the magnetization and the direction of the precession around the field \cite{LandauLifshitz1935}. The damping term will point in a direction as to attempt to align the magnetization with the field, \eqref{eq:mag_undamped} therefore becomes

\begin{align}
\label{eq:LL}
\frac{\textrm{d} \vec{M}}{\textrm{d} t} = -\gamma (\vec{M} \times \vec{H}_{eff} + \frac{\alpha}{M_s} \vec{M}\times(\vec{M}\times\vec{H}_{eff})),
\end{align}

with $\alpha > 0$ being a dimensionless damping constant. This is known as the Landau--Lifshitz equation.

\subsection{The Gilbert damping term}
The Landau-Lifshitz equation agrees with experiments for low damping constants $\alpha$, but it does not perform well when the damping constants get large. To improve the model of damping in \eqref{eq:LL}, Gilbert proposed a damping term on a different form. By comparing the damping of the magnetization to dampings in other physical systems, Gilbert used a Rayleigh dissipation functional that depended on the time derivative of the magnetization to model the dissipative force \cite{Gilbert2004Classics}. This is analogous to a model for friction where the Rayleigh dissipation functional depends on the time derivative of the position, which is just the velocity. By using this new damping term that is a function of $\frac{\textrm{d} \vec{M}}{\textrm{d} t}$, the Gilbert form of the Landau--Lifshitz equation becomes

\begin{align}
\label{eq:LLG_implicit}
\frac{\textrm{d} \vec{M}}{\textrm{d} t} = -\gamma \vec{M} \times (\vec{H}_{eff} - \eta \frac{\textrm{d} \vec{M}}{\textrm{d} t}),
\end{align}

with $\eta$ being a damping parameter. This equation is implicit in $\frac{\textrm{d} \vec{M}}{\textrm{d} t}$. It can be shown that \eqref{eq:LLG_implicit} and \eqref{eq:LL} are mathematically equivalent by rewriting it to an explicit form in $\frac{\textrm{d} \vec{M}}{\textrm{d} t}$. To rewrite \eqref{eq:LLG_implicit} we must find an expression for $\vec{M}\times\frac{\textrm{d} \vec{M}}{\textrm{d} t}$ that does not involve a cross-product. To do this we can just take the cross product of \eqref{eq:LLG_implicit} with $\vec{M}$ from the left:

\begin{align}
\label{eq:mtimesdmdt}
\vec{M}\times\frac{\textrm{d} \vec{M}}{\textrm{d} t} = -\gamma \vec{M}\times(\vec{M}\times\vec{H}_{eff}) + \gamma\eta \vec{M}\times(\vec{M}\times\frac{\textrm{d} \vec{M}}{\textrm{d} t}).
\end{align}

The first term on the right hand side is proportional to one of the terms in \eqref{eq:LL}, so we leave that be. The triple cross product involving $\frac{\textrm{d} \vec{M}}{\textrm{d} t}$ can be rewritten using the relation $\vec{A}\times(\vec{B}\times\vec{C}) = (\vec{A}\cdot\vec{C})\vec{B} - (\vec{A}\cdot\vec{B})\vec{C}$:

\begin{align}
\label{eq:mag_triple_crossproduct}
\vec{M}\times(\vec{M}\times\frac{\textrm{d} \vec{M}}{\textrm{d} t}) =  (\vec{M}\cdot\frac{\textrm{d} \vec{M}}{\textrm{d} t})\vec{M} - (\vec{M}\cdot\vec{M})\frac{\textrm{d} \vec{M}}{\textrm{d} t}.
\end{align}

The first term on the right hand side can be found by taking the scalar product of $\vec{M}$ and \eqref{eq:LLG_implicit}. It is easy to see that this is 0, as the entire right hand side of \eqref{eq:LLG_implicit} is perpendicular to $\vec{M}$. It is also worth noting that when $\vec{M}\cdot \frac{\textrm{d} \vec{M}}{\textrm{d} t} = 0$ the modulus $|\vec{M}| = M_s$ is a constant, as 

\begin{align}
\frac{\textrm{d}}{\textrm{d} t} (M_s^2) = \frac{\textrm{d}}{\textrm{d} t} (\vec{M}\cdot\vec{M}) = 2 \vec{M} \cdot \frac{\textrm{d} \vec{M}}{\textrm{d} t}.
\end{align}

This is in agreement with what we have discussed earlier. The second term on the right hand side in \eqref{eq:mag_triple_crossproduct} is just $-M_s^2 \frac{\textrm{d} \vec{M}}{\textrm{d} t}$. Combining the results of \eqref{eq:mtimesdmdt} and \eqref{eq:mag_triple_crossproduct} and inserting it into \eqref{eq:LLG_implicit}, we can simplify the expression to

\begin{align}
\label{eq:LLG_oldparam}
\frac{\textrm{d} \vec{M}}{\textrm{d} t} = -\frac{\gamma}{1 + \gamma^2\eta^2 M_s^2}(\vec{M}\times\vec{H}_{eff} + \gamma\eta \vec{M} \times (\vec{M}\times\vec{H}_{eff})).
\end{align}

Comparing this to the Landau--Lifshitz equation in \eqref{eq:LL} we see that by letting

\begin{align}
\eta &= \frac{\alpha}{\gamma M_s}, \\
\tilde{\gamma} &= \frac{\gamma}{1+\alpha^2},
\end{align}

\eqref{eq:LLG_oldparam} can be written on the form of the Landau--Lifshitz equation:

\begin{align}
\label{eq:LLG}
\frac{\textrm{d} \vec{M}}{\textrm{d} t} = -\tilde{\gamma} (\vec{M} \times \vec{H}_{eff} + \frac{\alpha}{M_s} \vec{M}\times(\vec{M}\times\vec{H}_{eff})).
\end{align}

This is the explicit Landau--Lifshitz--Gilbert equation. The difference between the Landau--Lifshitz and the Landau--Lifshitz--Gilbert equation is that the gyromagnetic ratio in the LLG equation is redefined to include the dimensionless damping parameter $\alpha$. When $\alpha \ll 1$, the Landau--Lifshitz equation is a good approximation of the LLG equation, but as the damping parameter becomes larger the correction in the gyromagnetic ratio becomes significant. The LLG equation agrees better with experiments where the damping constant $\alpha$ is large \cite{GilbertKelly1955}.

\section{Domain walls}
To minimize the energy a ferromagnet of a significant size will divide itself into domains. A domain is a region where the magnetization is uniform, and the domains in a magnet are separated by something called domain walls. These are regions where the magnetization in the material switches orientation between two domains with different direction in the magnetization. There are many different domain wall structures, but what they all have in common is that in the static case they have to satisfy \eqref{eq:LLG} with the left hand side being zero. It is easily seen that this is satisfied when the local magnetization is parallel to the effective field $\vec{H}_{eff}$ at that point. From the derivation of the effective field, we remember that $\vec{H}_{eff}$ and $\vec{M}$ were parallel when the system was at an equilibrium, meaning $\delta E = 0$. This will be an easier problem to solve, as we can write the energy in terms of scalar quantities such as the components of the magnetization. To study the nature of the domain walls we will at first only consider internal effects, and not consider the case where an external magnetic field is applied. A static domain wall is therefore a magnetization structure that minimizes the exchange energy, the anisotropic energy and the demagnetization energy simultaneousy.

\subsection{The charge avoidance principle}
In general, it's very hard to determine what configuration in the magnetization that will lead to a minimum in the demagnetization energy. For some magnetic particles we saw that it was possible to treat the demagnetization energy as an effective shape anisotropy, which makes the form of the energy easier to minimize, but this is not always possible. To make this task a little easier, it is useful to introduce magnetic charges. It is important to note that this is purely a mathematical concept, as magnetic monopoles have never been detected. This method is just a comparison to electric charges and fields. The Maxwell equations for the magnetic field in the magnetostatic limit are

\begin{align}
\nabla \times \vec{H} &= 0, \\
\nabla \cdot \vec{B} &= 0.
\end{align}

Due to the form of the first equation it is possible to write the magnetic field $\vec{H}$ as a gradient of a potential: $\vec{H} = -\nabla U_M$. Using the second equation and the relation between the magnetic induction $\vec{B}$ and magnetic field $\vec{H}$, 

\begin{align}
\vec{B} = \mu_0(\vec{H}+\vec{M}),
\end{align}

one ends up with the Poisson equation for the magnetic potential $U_M$:

\begin{align}
\nabla^2 U_M = \nabla \cdot \vec{M}.
\end{align}

Comparing this to the Poisson equation for the electric potential $U_E$,

\begin{align}
\nabla^2 U_E = -\frac{\rho_f}{\epsilon},
\end{align}

we see that $-\nabla \cdot \vec{M}$ behaves as a free magnetic charge density, or a volume charge density. We therefore define

\begin{align}
\label{eq:magchargedensity}
\rho_M = - \nabla \cdot \vec{M}.
\end{align}

At the boundary of a magnet there is a discontinuty in the magnetization $\vec{M}$, as it drops to zero outside the magnet. If we apply the divergence theorem to \eqref{eq:magchargedensity} where we integrate over a Gaussian pillbox that is infinitesimally thin and has one surface on the outside of the magnet and one on the inside, we find that

\begin{align*}
\int_{\Omega} \d V \rho_M  &= -\int_{\Omega} \d V \nabla \cdot \vec{M} \\
&= -\int_{\partial \Omega} \d {\vec{S}}\cdot\vec{M} \\
&= -A(\vec{M}_{out}\cdot\hat{n} - \vec{M}_{in} \cdot \hat{n}) \\
&= A \vec{M} \cdot\hat{n}.
\end{align*}

Here $A$ is the area of the surfaces of the pillbox parallel to the surface of the magnet, and $\hat{n}$ is the unit vector pointing out of the surface of the magnet. The integral of $\rho_M$ over the volume of the magnet is just the total magnetic charge $Q_M$, we can therefore define a magnetic surface charge density

\begin{align}
\label{eq:magsurfacecharge}
\sigma_M = \frac{Q_M}{A} = \vec{M}\cdot\hat{n}.
\end{align}

Both the magnetic volume charge density $\rho_M = -\nabla\cdot\vec{M}$ and surface charge density $\sigma_M = \vec{M}\cdot\hat{n}$ will act as sources in the magnetic potential $U_M$. To minimize the demagnetizing field $\vec{H}_D$ and thereby the demagnetization energy $E_D$, it is therefore desireable to avoid these magnetic charges in the material as much as possible. This is the charge avoidance principle \cite{Coey}. We see that physically this means that it costs energy if the magnetization of a magnet has a component perpendicular to its surface, or the divergence of the magnetization is different from zero inside the magnet.

\subsection{Bloch walls}
To simplify the calculation of the magnetization structure in domain walls we only consider the exchange and anisotropic energies. This has some merit as we can either assume that the demagnetization energy will be on the form of a uniaxial anisotropy, or we can evaluate how the solution does with regards to the charge avoidance principle.

The first case we consider is a 1D chain of magnetic moments along the $x$-axis, with an easy axis along the $z$-axis and the magnetization being parallel to the easy axis at $x = \pm \infty$. The energy then can be written as

\begin{align}
\label{eq:BlochEnergy}
E = \int \d V (\frac{A}{M_s^2}(\frac{\partial}{\partial x}\vec{M}(x))^2 - K \cos ^2 \theta (x)).
\end{align}

$K$ is strictly positive in this case. The magnetization vector can be written in terms of spherical coordinates, so that $\vec{M}(x) = M_s (\sin \theta (x) \cos \phi (x), \sin \theta (x) \sin \phi (x), \cos \theta (x))$. We see that if we let the $x$-component of $\vec{M}$ be zero, there will be no magnetic volume charges as $\nabla \cdot \vec{M}(x) = 0$. We therefore restrict the magnetization to be in the $yz$-plane by letting $\phi = \pi/2$. The first term in the integral can then be written as

\begin{align}
\frac{A}{M_s^2}(\frac{\partial}{\partial x}\vec{M}(\vec{r}))^2 = A(\cos\theta \frac{\partial \theta}{\partial x} \hat{y} - \sin\theta \frac{\partial \theta}{\partial x} \hat{z})^2 = A (\frac{\partial \theta}{\partial x})^2.
\end{align}

The energy then becomes

\begin{align}
E = \int \d A \int \d x (A (\frac{\partial \theta}{\partial x})^2- K \cos ^2 \theta (x)) = \int \d A \int \d x \epsilon(x, \theta, \frac{\partial \theta}{\partial x}).
\end{align}

For us to have equilibrium, $\delta E = 0$, the integrand $\epsilon$ must satisfy the Euler--Lagrange equation:

\begin{align}
\frac{\partial \epsilon}{\partial \theta} - \frac{\textrm{d}}{\textrm{d} x} \frac{\partial \epsilon}{\partial (\frac{\partial \theta}{\partial x})} = 0.
\end{align}

Plugging in $\epsilon$, we get the second order differential equation

\begin{align}
\label{eq:BlochWall}
K\sin \theta (x) \cos \theta (x) = A \frac{\partial^2 \theta}{\partial x^2}.
\end{align} 

From this equation one can see both the domain walls and the domains. The domains are regions where the magnetization is uniform, meaning $\theta$ is a constant. It can easily be seen that $\theta = n\pi/2$ ($n = 0, 1, 2, \ldots$) satisfies this equation, but one should note that only the solutions $\theta = n\pi$ are stable, as that is the solution that minimizes the anisotropic energy when we have chosen the easy axis to be in the $z$-direction. For the domain walls the magnetization is not uniform, meaning we also have to consider the differentiation term. To solve this equation we multiply \eqref{eq:BlochWall} by $\frac{\partial \theta}{\partial x}$,

\begin{align*}
K\sin \theta (x) \cos \theta (x) \frac{\partial \theta}{\partial x} = A \frac{\partial^2 \theta}{\partial x^2}\frac{\partial \theta}{\partial x},
\end{align*}

which is equivalent to

\begin{align}
\frac{\textrm{d}}{\textrm{d} x} (K \sin ^2 \theta - A (\frac{\partial \theta}{\partial x})^2) = 0.
\end{align}

The expression inside the parantheses must therefore be a constant, and to determine what that constant is we have to use some boundary conditions. We want to describe the domain wall between two domains, meaning the domains are located at $x = \pm \infty$. Inside the domains there is no variation in the magnetization, so $\theta$ is a constant. The stable value of $\theta$ is either 0 or $\pi$, as discussed earlier. This means that the expression inside the parantheses is 0. We can then integrate and make the equation a first order differential equation:

\begin{align*}
K \sin ^2 \theta(x) &= A (\frac{\partial \theta(x)}{\partial x})^2 \\
\implies \frac{\textrm{d} \theta}{\sin \theta} &= \pm \sqrt{\frac{K}{A}} \d x.
\end{align*}

For simplicity we choose the positive solution, which would correspond to $\theta = 0$ at $x = -\infty$. The negative solution will lead to the same answer as the positive, only with the domains switched, so that $\theta = \pi$ at $x = -\infty$. We note that both sides of the equation are dimensionless, and use that information to define a characteristic length $\delta_{DW}$ of the system, with

\begin{align}
\delta_{DW} = \sqrt{\frac{A}{K}}.
\end{align}

Using that

\begin{align*}
\int \frac{d\theta}{\sin\theta} = \ln \tan\frac{\theta}{2} + C,
\end{align*}

we find that

\begin{align}
\ln \tan\frac{\theta}{2} - \ln \tan\frac{\theta_0}{2} = \frac{x-x_0}{\delta_{DW}}.
\end{align}

To get the final expression for $\theta(x)$, we must apply a last condition. As we want to describe a domain wall, we choose the domains on the different sides of the wall to be different. This is not a necessity, as one can also have two parallel domains separated by a 360$^o$ domain wall. We've seen that the stable domains occur for $\theta = 0$ and $\theta = \pi$. It is then reasonable to assume that if the function $\theta (x)$ is smooth, it will take the value $\theta = \pi/2$ at some point in the domain wall. If we let that point be located at $x = 0$, we find that

\begin{align}
\label{eq:thetaBloch}
\theta(x) = 2\arctan(\exp(\frac{x}{\delta_{DW}})).
\end{align}

Using the relations

\begin{align*}
\cos(2\arctan(x)) &= \frac{1-x^2}{1+x^2}, \\
\sin(2\arctan(x)) &= \frac{2x}{1+x^2},
\end{align*}

it is easy to show that the magnetization components become

\begin{align}
M_z &= - M_s\tanh\frac{x}{\delta_{DW}}, \label{eq:BlochMagZ} \\
M_y &= \frac{M_s}{\cosh\frac{x}{\delta_{DW}}}.
\end{align}

\begin{figure}[h!]
\centering
\begin{subfigure}{.5\textwidth}
  \centering
  \includegraphics[width=1.0\linewidth]{Figures/BlochWallMz}
  \caption{}
\end{subfigure}%
\begin{subfigure}{.5\textwidth}
  \centering
  \includegraphics[width=1.0\linewidth]{Figures/BlochWallMy}
  \caption{}
\end{subfigure}
\caption{The magnetization component in the $z$-direction (a) and in the $y$-direction (b) inside a Bloch domain wall.}
\label{fig:BlochWall}
\end{figure}

The magnetization components are shown as a function of $x/\delta_{DW}$ in Figure \ref{fig:BlochWall}. This type of domain wall where the magnetization rotates out of the plane that is parallel to the magnetization in the domains is known as a Bloch wall \cite{Bloch1932}. In the derivation of the expression for the Bloch wall we chose this direction of rotation as this magnetization structure did not create any magnetic volume charges $-\nabla\cdot\vec{M}$. To minimize the demagnetization energy we want to avoid both magnetic volume and surface charges. We see that surface charges can be created by the Bloch domain wall if the plane the domains are in is parallel and close to the surface of the magnet. This is because the Bloch domain wall rotates out of that plane, giving a component that is normal to the magnet's surface. If the domain wall is far away from a surface, however, there will not be any creation of a surface charge. Bloch domain walls are the most common domain walls in bulk materials. This is because of the thickness of the material makes it more energetically favorable to avoid volume charges than surface charges.

\subsection{N\'{e}el walls}
If the magnetic material has a large surface area, such as thin films, it may be more energetically favorable to avoid surface charges than to avoid volume charges. To have a domain wall that does this, we must require that the magnetization in the domain wall rotates in the plane of the domains, unlike the Bloch domain wall where the magnetization rotates out of that plane. If we look at a similar example as in the derivation of the Bloch domain wall, we must have a magnetization that rotates in the $xz$-plane rather than the $yz$-plane, as we let the $y$-direction be the direction normal to the magnet's surface. The easy axis in the material is kept to be along the $z$-axis. The energy expressions and the equation for $\theta (x)$ are exactly the same as for the Bloch domain wall, the only difference is that instead of forcing $\phi$ to be $\pi/2$ to avoid magnetic volume charges, we force $\phi$ to be 0 to let the $M_y$ component be 0 and thereby avoiding generating magnetic surface charges. The function $\theta(x)$ is therefore exactly the same as for the Bloch domain wall, given by \eqref{eq:thetaBloch}. The magnetization vector changes, however, from being $\vec{M} = M_s(0, \sin\theta, \cos\theta)$ for the Bloch domain wall to being $\vec{M} = M_s(\sin\theta, 0, \cos\theta)$ for our current domain wall. In other words, the magnetization component in the $x$- and $y$-directions just switch. The magnetization therefore looks the same as in Figure \ref{fig:BlochWall}, with $M_y \rightarrow M_x$ and $M_y=0$. This type of domain wall where the magnetization rotates in the plane of the domain magnetizations is known as a N\'{e}el domain wall. This type of domain wall does not generate any surface charges, assuming the domains are parallel to the surface of the magnet, but generates magnetic volume charges as the divergence of the magnetization is not zero. We can calculate the magnetic volume charge that the N\'{e}el wall generates:

\begin{align*}
\rho_M &= -\nabla\cdot\vec{M}(x) \\
&= -M_s \frac{\partial}{\partial x} \sin \theta(x) \\
&= \frac{M_s}{\delta_{DW}}\frac{\tanh\frac{x}{\delta_{DW}}}{\cosh\frac{x}{\delta_{DW}}}.
\end{align*}

\begin{figure}[h!]
\begin{center}
\includegraphics[width=0.8\textwidth]{Figures/MagneticVolumeCharge.pdf} 
\caption{The distribution of the magnetic volume charge inside a N\'{e}el domain wall.}
\label{fig:magVolumeCharge} 
\end{center}
\end{figure}

This magnetic volume charge distribution is shown in Figure \ref{fig:magVolumeCharge}. It is antisymmetric, and the total magnetic volume charge is zero. This magnetic volume charge therefore acts as a magnetic dipole.

\subsection{Domain wall width and energy}
To find the width of the domain wall separating the two domains, we need to define where the domain begins, as $M_z$ only approaches $\pm M_s$ asymptotically. One way to do that is to find the value of $x$ where $M_z$ has reached a certain percentage of $M_s$, and then let the domain wall width be twice that value as the domain wall is antisymmetric around $x=0$. This can be done by inverting \eqref{eq:BlochMagZ}. We then find that the width of the domain wall $\Delta_{DW}$ is

\begin{align}
\Delta_{DW} = 2\delta_{DW}\ln\sqrt{\frac{M_s+|M_z|}{M_s-|M_z|}}.
\end{align}

If we let the domains begin when $|M_z| = 0.99\cdot M_s$, we get that the domain wall width is $\Delta_{DW} \approx 5.3 \cdot \delta_{DW}$. This is true for both the Bloch domain walls and N\'{e}el domain walls we have derived earlier. Remembering that the characteristic length $\delta_{DW}$ was defined as $\delta_{DW} = \sqrt{A/K}$, we see that the width of the domain wall is determined by what is the more dominating term in the energy; the exchange interaction or the anisotropy. If the exchange interaction influences the energy much more than the anisotropy, the domain wall becomes very wide, as the magnetization wants to change very slowly to reduce the exchange energy. If the material is isotropic, meaning $K\rightarrow0$, the width of the domain wall becomes infinite. This is because when there is no anisotropy the stable energy state is when all spins are aligned. Anisotropy is therefore necessary for the existence of domain walls in this model where the demagnetization energy is not explicitly accounted for.

Even though domain walls are stable states, they have an energy cost from the exchange interaction and the anisotropic energy compared to uniform domains. This energy cost can be found by first rewriting \eqref{eq:BlochEnergy} to remove the constant energy in the domains, and then plug in our domain wall solutions to do the integral over the domain wall. The integrand is $-K$ in our domains ($\theta = 0$ and $\theta = \pi$), so we add a constant term of $K$ to the integrand to let the domains have no energy contributions from anisotropy or the exchange interaction. We then integrate to find the energy per domain wall area:

\begin{align*}
\frac{\textrm{d} E_{DW}}{\textrm{d} A} &= \int_{-\infty}^{\infty} \d x (\frac{A}{M_s^2}(\frac{\partial}{\partial x}\vec{M}(x))^2 - K \cos ^2 \theta (x) + K) \\
&= \int_{-\infty}^{\infty} \d x (\frac{A}{\delta_{DW}^2}\frac{1}{\cosh^2\frac{x}{\delta_{DW}}} + K \frac{1}{\cosh^2\frac{x}{\delta_{DW}}}) \\
&= 4K\delta_{DW} = 4\sqrt{AK}.
\end{align*}

\section{Domain wall dynamics}
Domain walls can be moved by applying an electric current to the magnetic material. This is done either by a momentum transfer or spin transfer between the local electrons and the conduction electrons. A momentum transfer would correspond to actual movement of the electrons in the domain wall, so that the domain wall moves with them, while a spin transfer corresponds to a shift in the local magnetization due to the exchange interaction with the conduction electrons. Here only the domain wall moves, and the position of the electrons in the domain wall is unchanged. For domain walls where the magnetization doesn't change abruptly, so that the spin of the conduction electron can follow the local magnetization more or less adiabatically, the spin transfer effect will be the dominant cause of the domain wall motion \cite{KohnoTatara-04}. 

\subsection{Spin-transfer torque}
The adiabatic spin-transfer torque is a consequence of conservation of angular momentum, or spin. As the spin of the conduction electron follows the variation in the magnetization in the domain wall, there will be a torque acting on the local magnetization in the opposite direction of the rotation of the itinerant spin. This is simply Newton's third law; for the torque acting on the itinerant electron moving through the local magnetization, there will be an equal and opposite torque acting on the local magnetization. When this process is adiabatic the total spin of the local and itinerant electrons is conserved. There is also a non-adiabatic spin-transfer torque, which occurs when the itinerant electron does not move completely adiabatically through the magnet. 

To model the spin-transfer torque acting on the local magnetization in the magnet, we mainly follow the derivation by Zhang and Li \cite{ZhangLi-04}. The spin density of the itinerant electrons in the applied current can in general be written as

\begin{align}
\label{eq:mag_current}
\vec{m} = \langle \vec{s} \rangle = \frac{m_0}{M_s}\vec{M} + \delta\vec{m}.
\end{align}

Here we have split the magnetization into two components, the first of which is parallel to the local magnetization $\vec{M}$, and the second component $\delta\vec{m}$ which  is perpendicular to the local magnetization. It is assumed that the component parallel to the local magnetization is much larger than the perpendicular component, as the magnetization of the itinerant electrons follows the local magnetization mostly adiabatically. One should also note that $m_0/M_s \ll 1$ as the itinerant current is not near being saturated. The time evolution of the itinerant spins are given by the continuity equation

\begin{align}
\label{eq:spin_continuity}
\frac{\partial \vec{s}}{\partial t} + \nabla \hat{J} = \frac{1}{i\hbar} \left[ \vec{s}, H_{sd} \right] - \Gamma(\vec{s}),
\end{align}
with $\hat{J}$ being the spin current operator and $\Gamma(\vec{s})$ describes the relaxation of the non-adiabatic spins. The s-d interaction Hamiltonian $H_{sd}$ describes the exchange energy between the itinerant spin $\vec{s}$ and the local spin $\vec{S}$,

\begin{align}
H_{sd} = -J \vec{s} \cdot \vec{S} = -J \vec{s} \cdot (-\frac{S}{M_s} \vec{M}).
\end{align}

The commutator in \eqref{eq:spin_continuity} can be approximated by

\begin{align}
\frac{1}{i\hbar} \left[ \vec{s}, H_{sd} \right] = - \frac{1}{i\hbar} \left[ H_{sd}, \vec{s} \right] \approx -\frac{1}{i\hbar}\sum_{i, j} \frac{\textrm{d} H_{sd}}{\textrm{d} s_i}\left[ s_i, s_j \right]
\end{align}

in the semi-classical limit \citep{kruger2006current}. Using that the dimensionless spins can be written on the form $s_i = \sigma_i/2$, with $\sigma_i$ being the Pauli matrices, we use the commutation relation between the Pauli matrices to find that

\begin{align}
\left[ s_i, s_j \right] = \sum_k i\varepsilon_{ijk} s_k.
\end{align}

Combining this with $\frac{\textrm{d} H_{sd}}{\textrm{d} s_i} = -JS_i$, we get that

\begin{align}
\frac{1}{i\hbar} \left[ \vec{s}, H_{sd} \right] \approx - \frac{1}{i\hbar} \sum_{i,j, k} (-J S_i) i\varepsilon_{ijk}  s_k = -\frac{J}{\hbar} \vec{S} \times \vec{s}.
\end{align}

Using this result, and letting the operators become their expectation values, we get the continuity equation for $\vec{m}$:

\begin{align}
\label{eq:spindensity_continuity}
\frac{\partial \vec{m}}{\partial t} + \nabla \langle\hat{J}\rangle = -\frac{J S}{\hbar M_s} \vec{m} \times \vec{M} - \langle\Gamma(\vec{m})\rangle.
\end{align}

The first term on the right hand side is a torque acting on $\vec{m}$ from the interaction between $\vec{m}$ and $\vec{M}$. By Newton's third law there will be an equal and opposite torque acting on the local magnetization $\vec{M}$, so that the spin-transfer torque becomes

\begin{align}
\label{eq:STT}
\vec{T}_{STT} = -\frac{J S}{\hbar M_s} \vec{M} \times \vec{m} = -\frac{J S}{\hbar M_s} \vec{M} \times \delta\vec{m}
\end{align}

To determine the spin-transfer torque we must find an expression for $\delta\vec{m}$. This is done by solving \eqref{eq:spindensity_continuity}. To do that one has to make some basic assumptions. We assume that the relaxation of the non-adiabatic spins is proportional to the non-adiabatic component of $\vec{m}$. By letting the average spin-flip relaxation time be defined by $\tau_{sf}$, the last term in \eqref{eq:spindensity_continuity} becomes $\langle\Gamma(\vec{m})\rangle \approx \delta\vec{m}/\tau_{sf}$. 

The spin current density is a tensor of rank 2, defined by the vectors that describe the current flow of the electrons and the local spin. As the spin and magnetization of an electron are anti-parallel, the adiabatic spin current density can then be written as

\begin{align}
\langle \hat{J}_{\textrm{adiabatic}} \rangle = -\frac{\mu_B P}{e} \vec{j}_e \otimes \frac{\vec{M}}{M_s}.
\end{align}

The factor $P$ describes the portion of the conduction electrons that are polarized in the direction of the local electrons, and $e$ is the elementary charge. The factors $\mu_B$ and $e$ are included to get the correct units. If one assumes a uniform charge current density $\vec{j}_e$, and that the non-adiabatic spin current is negligible (which is alright if the domain wall width is much greater than the transport length scale, see \cite{ZhangLi-04}), the spin current term becomes

\begin{align}
\nabla \langle \hat{J} \rangle = \nabla (-\frac{\mu_B P}{e} \vec{j}_e \otimes \frac{\vec{M}}{M_s}) = -\frac{\mu_B P}{e M_s} (\vec{j}_e \cdot \nabla) \vec{M}.
\end{align}

Remembering that we have an adiabatic component of $\vec{m}$ that is much greater than the non-adiabatic component $\delta\vec{m}$, we ignore the term $\frac{\partial \delta\vec{m}}{\partial t}$, which is necessary to obtain an analytical solution of $\delta \vec{m}$. Combining all of these assumptions and inserting them into \eqref{eq:spindensity_continuity}, the equation becomes 

\begin{align}
\label{eq:dm_implicit}
\frac{m_0}{M_s}\frac{\partial \vec{M}}{\partial t} - \frac{\mu_B P}{e M_s} (\vec{j}_e \cdot \nabla) \vec{M} = -\frac{1}{\tau_{ex} M_s} \delta\vec{m} \times \vec{M} - \frac{\delta\vec{m}}{\tau_{sf}},
\end{align}

where we have introduced the spin exchange relaxation timescale $\tau_{ex} = \hbar/JS$. This equation can be solved explicitly for $\delta\vec{m}$ in the same manner that we transformed the LLG equation from the implicit form to the explicit Landau--Lifshitz form. To do that we took the cross product of the equation with $\vec{M}$ from the left. Doing that, we find that the term $\delta\vec{m}\times\vec{M}$ can be written linearly in $\delta\vec{m}$ as

\begin{align}
\delta\vec{m} \times \vec{M} = \tau_{sf} (\frac{m_0}{M_s} \vec{M} \times \frac{\partial \vec{M}}{\partial t} - \frac{\mu_B P}{e M_s} \vec{M}\times((\vec{j}_e\cdot\nabla)\vec{M}) + \frac{M_s}{\tau_{ex}} \delta\vec{m}),
\end{align}

as $\vec{M} \times (\delta\vec{m}\times\vec{M}) = M_s^2\delta\vec{M}$. Inserting this result back into \eqref{eq:dm_implicit} and solving for $\delta\vec{m}$, we find the explicit expression

\begin{align}
\label{eq:dm_explicit}
\delta\vec{m} = \frac{\tau_{ex}}{1+\xi^2}(-\frac{m_0\xi}{M_s}\frac{\partial \vec{M}}{\partial t} + \frac{\xi \mu_B P}{e M_s}(\vec{j}_e\cdot\nabla)\vec{M} - \frac{m_0}{M_s^2}\vec{M}\times \frac{\partial \vec{M}}{\partial t} + \frac{\mu_B P}{e M_s^2}\vec{M}\times((\vec{j}_e\cdot\nabla)\vec{M})),
\end{align}

with $\xi$ being defined as the ratio $\xi = \tau_{ex}/\tau_{sf}$. Inserting this into the expression for the spin-transfer torque given in \eqref{eq:STT}, we find that the spin-transfer torque $\vec{T}_{STT}$ as a function of the local magnetization $\vec{M}(\vec{r}, t)$ and the charge current density $\vec{j}_e$ is 

\begin{align}
\label{eq:STT_final}
\vec{T}_{STT} = \frac{1}{1+\xi^2}(\frac{m_0\xi}{M_s}\vec{M}\times\frac{\partial \vec{M}}{\partial t} - \frac{\xi\mu_B P}{e M_s^2}\vec{M}\times(\vec{j}_e\cdot\nabla)\vec{M} - \frac{m_0}{M_s} \frac{\partial \vec{M}}{\partial t} - \frac{\mu_B P}{e M_s^3} \vec{M}\times (\vec{M}\times(\vec{j}_e\cdot\nabla)\vec{M})).
\end{align}

Here we have used that $\vec{M}\times(\vec{M}\times\frac{\partial \vec{M}}{\partial t}) = -M_s^2\frac{\partial \vec{M}}{\partial t}$ as $\vec{M}\cdot\frac{\partial \vec{M}}{\partial t} = 0$, and that $\tau_{ex} = \hbar/JS$. Using this result we can extend the LLG equation in \eqref{eq:LLG_implicit} to also include the spin-transfer torque effects from the charge current, by inserting $\vec{T}_{STT}$ into the right hand side. Looking at the implicit LLG equation we see that we already have terms that go as $\frac{\partial \vec{M}}{\partial t}$ and $\vec{M}\times\frac{\partial \vec{M}}{\partial t}$. These terms in the spin-transfer torque $\vec{T}_{STT}$ can therefore be incorporated into the LLG equation by redfining the gyromagnetic ratio $\gamma$ and the damping parameter $\eta$. If one inserts the terms in $\vec{T}_{STT}$ that do not depend on the current into \eqref{eq:LLG_implicit}, one finds that by letting

\begin{align}
\beta &= \frac{m_0}{M_s}\frac{1}{1+\xi^2}, \\
\gamma ' &= \frac{\gamma}{1+\beta}, \\
\eta ' &= \eta + \frac{\xi\beta}{\gamma},
\end{align}

the LLG equation keeps its original form. As $m_0\ll M_s$, we also get $\beta \ll 1$, and so the corrections to $\gamma$ and $\eta$ are relatively small. The real impact from the spin-transfer torque comes from the terms dependent on the current. Including these terms the extended LLG equation becomes

\begin{align}
\label{eq:LLG_current}
\frac{\textrm{d} \vec{M}}{\textrm{d} t} = -\gamma ' \vec{M} \times (\vec{H}_{eff} - \eta '\frac{\textrm{d} \vec{M}}{\textrm{d} t}) - \frac{b_J}{M_s^2} \vec{M}\times (\vec{M}\times(\hat{j}_e\cdot\nabla)\vec{M})) - \frac{c_J}{ M_s}\vec{M}\times(\hat{j}_e\cdot\nabla)\vec{M},
\end{align}

with $\hat{j}_e$ being the unit vector along the electring current, and $b_J$ and $c_J$ being defined by

\begin{align}
b_J &= \frac{1}{1+\xi^2} \frac{\mu_B j_e P}{e M_s}, \\
c_J &= \xi b_J.
\end{align}

By looking at the directions of the $b_J$ and $c_J$ terms in \eqref{eq:LLG_current}, one sees that the $b_J$ term is parallel to the direction along which the local magnetization varies, and that the $c_J$ term is perpendicular to the plane that the local magnetization and the change in the local magnetization constitute. If one considers the magnetization of a N\'{e}el wall in the $x$-direction and a flow of electrons in the positive $x$-direction, the direction of the charge current becomes $\hat{j}_e = -\hat{x}$. The minus sign is because there is a flow of negative charges. If one lets the magnetization of the N\'{e}el wall point along $\hat{z}$ at $x = -\infty$, and along $-\hat{z}$ at $x = \infty$, and let the magnetization of the conduction electrons follow the local magnetization mostly adiabatically, the $b_J$ term will point in the opposite direction of $\frac{\partial \vec{M}}{\partial x}$ and the $c_J$ term will point along $-\hat{y}$. The $b_J$ term is therefore known as the adiabatic spin-transfer torque as it adjusts the local magnetization in the opposite direction of the adiabatic motion of the magnetization of the conduction electrons. Similarly the $c_J$ term is known as the non-adiabatic spin-transfer torque as it adjusts the local magnetization perpendicularly to the adiabatic motion of the magnetization of the conduction electrons. 

The form of the extended LLG equation in \eqref{eq:LLG_current} is implicit in $\frac{\textrm{d} \vec{M}}{\textrm{d} t}$. Like the implicit LLG equation, it can be transformed into an explicit form by taking the cross product of it with $\vec{M}$, to get a linear expression for $\vec{M}\times\frac{\textrm{d} \vec{M}}{\textrm{d} t}$ in $\frac{\textrm{d} \vec{M}}{\textrm{d} t}$. It is then easy to show that by choosing

\begin{align}
\tilde{\eta '} &= \frac{\alpha}{\gamma ' M_s}, \\
\tilde{\gamma '} &= \frac{\gamma '}{1+\alpha^2},
\end{align}

with $\alpha$ being a dimensionless damping constant like in the Landau--Lifshitz equation, we get the explicit form of the extended LLG equation:

\begin{align}
\nonumber \frac{\textrm{d} \vec{M}}{\textrm{d} t} = &-\tilde{\gamma '} \vec{M} \times  (\vec{H}_{eff} + \frac{\alpha}{M_s} \vec{M}\times \vec{H}_{eff}) \\ 
&- \frac{(1+\alpha\xi) b_J}{M_s^2(1+\alpha^2)} \vec{M}\times (\vec{M}\times(\hat{j}_e\cdot\nabla)\vec{M})) - \frac{(\xi-\alpha)b_J}{ M_s(1+\alpha^2)}\vec{M}\times(\hat{j}_e\cdot\nabla)\vec{M}.
\label{eq:LLG_current_explicit}
\end{align}

From this equation one can easily see that there will always be an adiabatic spin-transfer torque when a charge current is propagating through a domain wall, but there will only be a non-adiabatic spin-transfer torque if $\xi \neq \alpha$. As $\xi = \tau_{ex}/\tau_{sf}$, with $\tau_{ex}$ being a time scale in the exchange interaction and $\tau_{sf}$ being an average relaxation time of the non-adiabatic electrons, $\xi$ describes a dimensionless damping parameter for the conduction electrons. There will therefore be a non-adiabatic spin-transfer torque when the damping of the local and the conduction electron spins are different. The non-adiabatic spin-transfer torque causes a rotation of the domain wall, as one of the spins will relax faster than the other.


\subsection{Domain wall pinning}

\subsection{The motion of domain walls}

\section{Skyrmions}

\section{References}
\bibliography{main}
\bibliographystyle{unsrt}


\end{document}