\subsection{Anisotropic energy}
Magnetic materials may be anisotropic, meaning that certain directions of the magnetization in the material are more energetically favorable than others. This stands in contrast to isotropic materials where the energy is independent of the direction of the magnetization, assuming uniform magnetization. There are several possible causes for anisotropy in a material, the most common one being the crystalline structure in the material. The shape of the magnetic particles may also cause anistropy in the material. In anisotropic magnetic materials one defines different axes; the easy, intermediate and hard axes. If the magnetization is oriented along the easy axis, the anisotropic energy is at its minimum value. Similarly, if the magnetization is oriented along the hard axis the anisotropic energy is at its maximum value. The intermediate axis is the direction along which there is a saddlepoint in the anisotropic energy. In most materials the anisotropic energy is the same for both possible orientations of the magnetization along the axes. When this is the case, the terms in the anisotropic energy can only contain even powers of the components in the magnetization vector.

\subsubsection{Uniaxial anisotropy}
The simplest form of anisotropy is the case where there is only one axis that the anisotropic energy varies with. This is known as uniaxial anisotropy. This can often be found in crystal structures with a single principal axis, such as the hexagonal and simple tetragonal lattices. If we consider a uniform magnetization $\vec{M} = \left[M_x, M_y, M_z\right]$ and let one of the directions along the axis in the material that the anisotropical energy varies be the unit vector $\vec{e}$, the anisotrpic energy must be on the form

\begin{align}
E_A = V \sum_{i=0}^{\infty} K_i (\frac{\vec{M}\cdot\vec{e}}{M_s})^{2i}
\end{align}

if there is to be no preferential direction with respect to energy along the axis. Here $V$ is the volume, $M_s = |\vec{M}|$ the saturation magnetization and $K_i$ are the anisotropy constants. By only considering the lowest order in the anisotropical energy it becomes

\begin{align}
E_A = K V (\frac{\vec{M}\cdot\vec{e}}{M_s})^2
\end{align}

by ignoring the constant energy term as that only leads to a constant shift in the energy. If $K < 0$ the axis that $\vec{e}$ points along is then the easy axis, and if $K > 0$ the axis is the hard axis. Note that if we have a non-uniform magnetization we let

\begin{align*}
\vec{M} &\rightarrow \vec{M}(\vec{r}), \\
V &\rightarrow \int \d V.
\end{align*}

\subsubsection{Cubic anisotropy}
Many magnetic materials, such as iron, have cubic crystal structures. This can give rise to cubic anisotropy in the magnetic material. To describe cubic anisotropy it can be useful to introduce directional cosines that give the component of a unitary vector along the $x$-, $y$- and $z$-axes. Using spherical coordinates, these directional cosines are defined to be $\alpha = \sin\theta\cos\phi$, $\beta = \sin\theta\sin\phi$, $\gamma = \cos\theta$. Note that the directional cosines can also be written in terms of the magnetization, so that $\alpha = M_x/M_s$, $\beta = M_y/M_s$ and $\gamma = M_z/M_s$. If there is to be no preferential direction with respect to energy along the crystal axes, the anisotropic energy can only contain even powers of $\alpha$, $\beta$ and $\gamma$. In addition, to have cubic symmetry in the anisotropic energy, the energy must be invariant under any interchanges among the directional cosines \cite{Kittel:ISSP}. The first term in the anisotropical energy that depends on the directional cosines will therefore be proportional to $\alpha^2+\beta^2+\gamma^2$, but using the definitions of the directional cosines we find that

\begin{align}
\label{eq:cosinesconstraint}
\alpha^2+\beta^2+\gamma^2 = \frac{M_x^2+M_y^2+M_z^2}{M_s^2} = \frac{|\vec{M}|^2}{|\vec{M}|^2} = 1.
\end{align}

This term is therefore just a constant, so to find a term that varies with the directional cosines we must go to fourth order:

\begin{align}
\label{eq:cubicanisotropy}
E_A = E_A^{(0)} + \int (K_1 (\alpha^2\beta^2+\alpha^2\gamma^2+\beta^2\gamma^2) + \ldots ) \d V.
\end{align}
The sixth order term would be on the form $K_2 V \alpha^2\beta^2\gamma^2$ and so on. If one looks at the form of the energy in \eqref{eq:cubicanisotropy} one can determine the easy and hard axes in the material. The integrand has to be minimized along the easy axes, and maximized along the hard axes. If one only considers the fourth order terms in the energy, we have to minimize and maximize the function $K_1 (\alpha^2\beta^2+\alpha^2\gamma^2+\beta^2\gamma^2)$ with the constraint given by \eqref{eq:cosinesconstraint}. Using Lagrange multipliers, this breaks down to solving 

\begin{align}
\nabla f(\alpha, \beta, \gamma) = \lambda \nabla g(\alpha, \beta, \gamma),
\end{align}
with
\begin{align}
f(\alpha, \beta, \gamma) &= K_1 (\alpha^2\beta^2+\alpha^2\gamma^2+\beta^2\gamma^2), \\
g(\alpha, \beta, \gamma) &= \alpha^2+\beta^2+\gamma^2 - 1,\\
\nabla &= (\frac{\partial}{\partial\alpha}, \frac{\partial}{\partial\beta}, \frac{\partial}{\partial\gamma}).
\end{align}
This gives us the following set of equations to solve:

\begin{align}
2K_1\alpha (\beta^2+\gamma^2) &= 2\alpha\lambda, \\ 
2K_1\beta (\alpha^2+\gamma^2) &= 2\beta\lambda, \\ 
2K_1\gamma (\alpha^2+\beta^2) &= 2\gamma\lambda, \\ 
\alpha^2+\beta^2+\gamma^2 - 1 &= 0.
\end{align}

One can easily see that one solution of the equations is given by $\lambda = 0$, one of the directional cosines being $\pm 1$ and the other two being 0. Inserting this solution into $f(\alpha, \beta, \gamma)$ we get $f = 0$. \\

Assuming only one of the directional cosines is 0, for example $\gamma$, one gets

\begin{align}
K_1 \beta^2 &= \lambda, \\ 
K_1 \alpha^2 &= \lambda.
\end{align}

Using the constraint in \eqref{eq:cosinesconstraint} one finds $\lambda = K_1/4$, and both non-zero directional cosines being $\pm 1/2$. Inserting this solution into $f(\alpha, \beta, \gamma)$ we get $f = K_1/4$. \\

Assuming none of the directional cosines are 0, one gets

\begin{align}
K_1 (\beta^2+\gamma^2) &= \lambda, \\ 
K_1 (\alpha^2+\gamma^2) &= \lambda, \\ 
K_1 (\alpha^2+\beta^2) &= \lambda.
\end{align}

Using the constraint in \eqref{eq:cosinesconstraint} one finds $\lambda = 2K_1/3$, and each of the directional cosines being $\pm 1/\sqrt{3}$. Inserting this solution into $f(\alpha, \beta, \gamma)$ we get $f = K_1/3$. This means that if $K_1 > 0$ the energy is minimized by the first solution, making the $x$-, $y$- and $z$-axes the easy axes. The hard axes are then given by the third solution which corresponds to the four axes $x = \pm y = \pm z$, $x = \pm y = \mp z$. If $K_1 < 0$ the energy is maximized by the first solution and minimized by the third, so that the easy and hard axes switch from the case when $K_1 > 0$. The second solution denotes a local minimum if $K_1<0$ and a local maximum if $K_1>0$. The second solution does not give any intermediate axes, as there is no saddle point in the energy.

\subsection{Demagnetization energy}
The magnetic dipoles inside a ferromagnet will create a magnetic field, known as the demagnetization field. The energy in a ferromagnet will depend on the orientation of the magnetic dipoles with respect to the demagnetization field created by the other magnetic dipoles. This demagnetization energy is given by

\begin{align}
E_D = -\frac{1}{2}\int \d {^3}r \vec{M}(\vec{r})\cdot\vec{H}_D(\vec{r}),
\end{align}

with $\vec{H}_D$ being the demagnetization field. To determine the demagnetization energy one needs to find the magnetic field generated by a magnetic dipole $\vec{M}$. The vector potential generated by a magnetic dipole is known to be

\begin{align}
\vec{A}(\vec{r}) = \frac{\mu_0}{4\pi}\frac{\vec{M}\times\vec{r}}{r^3}.
\end{align} 

The magnetic induction $\vec{B}(\vec{r})$ is defined as

\begin{align}
\vec{B}(\vec{r}) = \nabla \times \vec{A}(\vec{r}).
\end{align}

Using the Einstein summation convention and the Levi--Civita tensor $\epsilon_{ijk}$ one can then find the magnetic induction generated by a magnetic dipole:

\begin{align*}
B_i &= \epsilon_{ijk} \partial_j A_k \\
&= \epsilon_{kij}\epsilon_{klm} \frac{\mu_0}{4\pi}\partial_j\frac{M_l r_m}{r^3} \\
&= (\delta_{il}\delta_{jm}-\delta_{im}\delta_{jl})\frac{\mu_0}{4\pi}(\frac{M_l\delta_{jm}}{r^3}-\frac{3M_lr_mr_j}{r^5}) \\
&= \frac{\mu_0}{4\pi}(\frac{3M_jr_jr_i}{r^5}-\frac{M_i}{r^3}).
\end{align*}

This means that the magnetic field strength of the demagnetization field at a position $\vec{r}$ is

\begin{align}
\label{eq:demagfield}
\vec{H}_D = \frac{1}{4\pi} \int \d {^3}r' (\frac{3(\vec{M}(\vec{r'}) \cdot (\vec{r}-\vec{r'})) (\vec{r}-\vec{r'})}{|\vec{r}-\vec{r'}|^5}-\frac{\vec{M} (\vec{r'})}{|\vec{r}-\vec{r'}|^3}).
\end{align}